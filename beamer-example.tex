\documentclass{beamer}

\usepackage[utf8]{inputenc}

\usepackage{textcomp} % For \texttildelow

\usepackage{lstcustom}


\title{The \texttt{lstcustom} package \& Beamer}
\author{Mark Royer}
%\institute{}
\date{2021}


\begin{document}

\frame{\titlepage}

\begin{frame}[fragile]
  \frametitle{Beamer Example}
  
  The following slides demo the \texttt{lstcustom} package with Beamer

  \begin{alertblock}{Important!}
    \begin{itemize}
    \item Frames that contain code must use the \texttt{fragile}
      option
    \item You can use the following delimiters with
      \texttt{\textbackslash lstinline}:\newline \$ , !  , | ,
      \texttildelow , + , - , \verb=^= , /
    \end{itemize}

  \end{alertblock}
  
\end{frame}


\begin{frame}[fragile]
  \frametitle{Inline Listing}
  
  % This slide is to check the \lstinline delimiters
  The 8 basic types in Java are
  \begin{itemize}
  \item \lstinline$boolean$,
  \item \lstinline!byte!,
  \item \lstinline|char|,
  \item \lstinline~short~,
  \item \lstinline+int+,
  \item \lstinline-long-,
  \item \lstinline^float^,
  \item and \lstinline/double/.
  \end{itemize}

\end{frame}


\begin{frame}[fragile]
  \frametitle{Block Listing}

  A slightly longer block example is shown in listing \ref{lst:simple}.

\begin{lstlisting}[caption={A simple listing.}, label={lst:simple}]
  /**
    * @param args
    *             Program arguments
    */
  public static void main(String[] args) {
    // Now for the enlightening message.
    System.out.println("Hmm... hello big world!";
    // TODO Finish this example
    // Just some comment that is probably too long to fit in the space provided....
  }
\end{lstlisting}

\end{frame}
  
\begin{frame}[fragile]
  \frametitle{File Listing Example}

  Listing \ref{lst:file} directly imports code from a \texttt{.java} file. Use
  \texttt{firstline} and \texttt{lastline} to select a region from the
  file.

  \lstinputlisting[firstline=32,lastline=40,
  caption={Lines 32-40 of the file \texttt{Rectangle.java}},
  label={lst:file}]
  {Rectangle.java}
  
\end{frame}


\begin{frame}[fragile]
  \frametitle{Customizations}

  \section*{}

  Sometimes you just have to do things yourself.
  For example, the enum below has no color or emphasis as is typical in Eclipse. 

\begin{lstlisting}[caption={Without manual customizations.}, label={lst:nocustom}]
  enum {
    FIRST,
    SECOND,
    THIRD
  }
\end{lstlisting}

\end{frame}


\begin{frame}[fragile]
  \frametitle{Customizations}

  You may add your own latex by escaping out of the listing.  I've added
  two macros to mimic Eclipse field (\texttt{\textbackslash ef}) and
  Eclipse field italicized (\texttt{\textbackslash efi}).  There is also
  a similar \texttt{\textbackslash ecom} and \texttt{\textbackslash
    ecomi} for comment styling. The results are shown below.


\begin{lstlisting}[caption={With manual customizations.}, label={lst:custom}]
  enum {
    (*@\efi{FIRST}@*),
    (*@\efi{SECOND}@*),
    (*@\efi{THIRD}@*)
  }
\end{lstlisting}

\end{frame}


\begin{frame}[fragile]
  \frametitle{Further Customizations}

  The nice thing about this approach is that you can use anything that
  you normally can in \LaTeX.  For example,


\begin{lstlisting}[caption={Something a little different.}, label={lst:wacky}]
  enum {
    (*@\efi{$\alpha$}@*),
    (*@\efi{$\beta$}@*),
    (*@\efi{$\gamma$}@*),
    (*@\efi{$e$}@*) // (*@\ecom{$\lim_{n\to\infty}{\left( 1 + \frac{1}{n}
        \right)^n}$}@*)
  }
\end{lstlisting}

  \noindent
  makes heavy use of the math environment. Of course, some of the
  lines are a little too big, so the bounding box appears slightly broken.

  
\end{frame}


\end{document}

% LocalWords:  lstcustom Beamer boolean firstline lastline enum ef
% LocalWords:  Customizations lstinline efi ecom ecomi
